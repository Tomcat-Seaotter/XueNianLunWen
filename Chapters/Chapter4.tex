\clearpage
\section{宁波居民区散点分布模型研究同城配送的模式和特性}
由于顾客多是散点状分布,这里涉及分组离散的网络图论问题。在确定派送策略时,我们可以通过合理的假设,将其转化为MTSP问题,因MTSP问题求解难度较大,可把MTSP问题转化为TSP问题解决,即可求出总的最短送货路程。以此最短路程为限制条件,利用基于最小生成树的深度优先搜索算法寻找合适的运行路线,并结合实际情况中的载重限制,对找出的运行路线进行调整修正,即可得出符合要求的运行路线。 
(下面利用TSP的蚁群算法的相关理论对问题进行分析。)

\subsection{MTSP算法模型}
给定n个城市(V1,V2,…,Vn),m名旅行商皆以城市V0为基地。令Wij表示 Vi城到Vj城的距离。
 ,指一条巡回路线。  为 巡回的总路程。目标是选择 m 条巡回路线使总路程最小,目标函数如下: 
 
(7)
当   时,称为对称型 MTSP。 
目标函数:以总的运行公里数最短为目标 
 
(8)
(约束条件:最短路径的约束)

\subsection{优化的TSP算法模型求解}
蚁群算法中每只蚂蚁作为一个简单的智能体具有以下特征:蚂蚁按照概率选择下一个要访问的城市,这一概率是要选择城市与当前城市之间距离以及信息素浓度的函数;为了让蚂蚁得到正确的解,已经访问过的城市在解构建完成之前是不能再被访问的,因此每个蚂蚁拥有一个禁忌表来存放已经访问过的城市;当蚂蚁完成解构建之后,会在访问过的边上释放信息素。
n个顶点,m个旅行商的MTSP问题可以转化为n'=n+m-1个顶点的TSP问题求解。扩大的(m-1)个顶点被称为人造顶点,其距离矩阵 转化为矩阵 ,将原问题矩阵:
 
转化后建立的TSP模型如下:
目标函数:
 
(9)
约束条件:
 
(10)

\subsection{快递区域配送应用}
某快递公司,假定所有快件在早上7点钟到达,早上9 点钟开始派送,要求与当天17点之前必须派送完毕,每个业务员每天平均工作时间不超过6小时,在每个送货点停留的时间为10分钟,途中速度为25km/h,每次出发最多能带25千克的重量。为了计算方便,将快件一律用重量来衡量,平均每天收到总重量为固定种类,并且假设街道平行于坐标轴方向。
那么需要计算出的合理派送策略需要作如下假设:
1、公司总部每次发货的对象是无差别的; 
2、尽量满足每条路线负重总量均衡; 
3、业务员行走拐弯的时间,路上的意外事故的耽搁时间忽略; 
4、街道平行于坐标轴方向。 
显然这是一个多旅行商问题,可将其转化为单旅行商问题,而对于 TSP问题利用 LINGO 软件利用蚁群算法求解,解得最短总路程为 492 公里。

表4.2 线路运行情况
路线编号	路线	路程(公里)	负重	时间(小时)
线1	0-16-17-28-29-0	90	23.4	4.27
线2	0-2-5-15-18-0	56	24.8	2.91
线3	0-4-24-25-0	68	22.7	3.22
线4	0-6-7-14-26-0	74	21	3.63
线5	0-19-30-23-21-0	92	20.6	4.35
线6	0-3-13-27-20-0	68	27.2	3.39
线7	0-8-12-11-0	46	19.1	2.34
线8	0-1-9-10-22-0	46	22.7	2.51

但根据工作时间均衡的原则,分别将路线 2 和路线8 合并,路线 6 和路线 7 合并,合并后每条路线由一个业务员送货,因此只需要6个业务员。最终调整后的每个业务员的运行路线如表 3 和图 2 所示,业务员的总运行路程为 546 公里。由于派送过程中有负重的约束条件,使得每个业务员的工作时间较短,多在4 个小时左右,相较于每个业务员最长工作时间6小时,表面上看利用率不足。由此快递局部在散点下派送的通过优化模型的方法,继而得到近似解。
对配送员的配送路径进行优化,建立合理的模型假设、设定模型参数,考虑配送任务中取货点和送货点配对有序,将取货路径和送货路径结合起来,建立同城配送取送货路径的联合优化模型,使用TSP遗传算法求解,得到总配送成本最小的最佳配送路径。
