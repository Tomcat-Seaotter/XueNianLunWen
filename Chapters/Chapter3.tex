\clearpage
\section{模糊聚类算法划分居民区域}

\subsection{区域划分聚类步骤}

为提高配送站点的配送效率,前提是要合理划分同城配送区域,确定最集中的配送中心点。
(1)选择k个初始中心点,例如c[0]=data[0],…c[k-1]=data[k-1];
(2)对于data[0]….data[n], 分别与c[0]…c[k-1]比较,假定与c[i]差值最少,就标记为i;
(3)对于所有标记为i点,重新计算c[i]={所有标记为i的data[j]之和}/标记为i的个数;
(4)重复(2)(3),直到所有c[i]值的变化小于给定阈值。
K-means算法简单易行、便于理解和操作,不但能够处理大规模的聚类问题,对本文研究的同城配送区域划分也有很好的应用效果。算法的聚类数k可以根据经验设定,也可以根据聚类分析对象的数据特征来设置。根据初次聚类的结果进行迭代更新,通过不断的迭代使结果不断接近判断标准,最终聚类结果达到目标标准,聚类中心不再发生变化,算法停止输出结果。

 
图3-1 居民区的散点分布经经纬度标准化后的散点图


\subsection{居民点的区域划分模型}

本文将案例中需求点的经纬度信息转化为平面坐标,便于使用K-means算法进行求解。用需求点之间距离的大小代表不同需求点之间的相似性,距离越小代表相似性越大。将不同需求点之间的相似性作为划分配送区域的依据,将相似性大的需求点划分到同一个配送区域。为方便后续使用遗传算法对配送路径优化模型进行求解,将地理位置的经纬度信息转化为平面坐标。
 

图3-1 K-means算法流程

算法的运行时间随着节点数量的变大而不断减小,这说明对模糊K-means的并行化切实提升了算法的运行速度,算法具有较好的扩展性。通过模糊K-means算法根据需求点的实际位置对配送区域进行划分,对聚类效果进行评价确定最佳聚类数为n,将k值设定为n,将配送区域化分为若干部分。

 
图3-2 K-means算法求解最佳配送区域过程

采用模糊K-means聚类算法,根据订单需求点的地理位置,确定最佳聚类数,将订单划分到不同的配送区域,得到满足实际配送要求的最佳聚类结果。

\subsection{居民点的区域可视化}
根据上面所述,对区县中的居民点的散点进行统一整合。如下图3-1宁波市各个区县级的地域分布图。其中图中红色、蓝色、绿色区域分别表示鄞州区、海曙区和江北区,灰色区域表示其他区县地区。
 
图3-1 宁波地理位置分布图

根据数据进行统计对区县中的居民点的散点进行分析地址统计,整合出大概地址后,可在地图上二维呈现出散点图状。如下图3-2宁波市中心部分居民区散点分布图。其中由于小区居民散点过多,显示的红点为主要的大集中居民区点。

 
图3-2宁波市中心部分居民区散点分布图

同时将居民点位置提取出来后,对居民点的具体地址的经纬度进行标准化等处理后,将便可对居民区的位置进行后续分析。
 
4宁波居民区散点分布模型研究同




\begin{table}[h]
	\caption{排序算法对比}
	\centering
	\begin{tabular}{||c|c|c|c|c|c||}
		\hline
		\multirow{2}*{类别}   &\multirow{2}*{排序方法} &\multicolumn{3}{|c|}{时间复杂度} &\multirow{2}*{稳定性}\\
		\cline{3-5}
		& &平均情况&最好情况&最坏情况& \\
		\hline
		\multirow{2}*{插入排序}&直接插入&$O(n^2)$&$O(n)$&$O(n^2)$&稳定\\
		\cline{2-6}
		&Shell排序&$O(n^{1.3})$&$O(n)$&$O(n^2)$&不稳定\\
		\hline
		\multirow{2}*{选择排序}&直接选择&$O(n^2)$&$O(n^2)$&$O(n^2)$&不稳定\\
		\cline{2-6}
		&堆排序&$O(n\log_{2}n)$&$O(n\log_{2}n)$&$O(n\log_{2}n)$&不稳定\\
		\hline
		\multirow{2}*{交换排序}&冒泡排序&$O(n^2)$&$O(n)$&$O(n^2)$&稳定\\
		\cline{2-6}
		&快速排序&$O(n\log_{2}n)$&$O(n\log_{2}n)$&$O(n^2)$& 不稳定\\
		\hline
		\multicolumn{2}{||c|}{归并排序}&$O(n\log_{2}n)$&$O(n\log_{2}n)$&$O(n\log_{2}n)$&稳定\\
		\hline
		\multicolumn{2}{||c|}{基数排序}&$d(r+n)$&$d(n+rd)$&$d(r+n)$& 稳定\\
		\hline
	\end{tabular}
\end{table}